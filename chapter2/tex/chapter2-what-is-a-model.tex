% !TeX encoding = utf8
\documentclass[aspectratio=169]{beamer}

\usepackage[english]{babel}
\usepackage{amsmath}
\usepackage{amssymb}
%\usepackage{mathpazo}
%\usepackage{palatino}
%\usefonttheme{professionalfonts}
%\renewcommand\familydefault{\rmdefault}
\usefonttheme[onlymath]{serif}

\DeclareMathOperator*{\E}{\mathbf{E}}
\DeclareMathOperator*{\Prob}{\mathbf{P}}
\newcommand{\A}{\mathbf{A}}
\newcommand{\Pro}{\mathbf{P}}
\newcommand{\xx}{\mathbf{x}}
\newcommand{\bb}{\mathbf{b}}
\newcommand{\ee}{\mathbf{e}}
\newcommand{\pp}{\mathbf{p}}
\newcommand\inner[2]{\langle #1, #2 \rangle }
\newcommand{\R}{\mathbb{R}}

\DeclareMathOperator*{\argmax}{arg\,max}
\DeclareMathOperator*{\argmin}{arg\,min}

\addtobeamertemplate{navigation symbols}{}{%
	\usebeamerfont{footline}%
	\usebeamercolor[fg]{footline}%
	\hspace{1em}%
	\insertframenumber/\inserttotalframenumber
}

\title{Hands on Machine Learning}
\subtitle{Chater 2 -- What is a Model?}
\author{\href{mailto:zoennchen.benedikt@hm.edu}{\textbf{Benedikt Z\"onnchen}}} 
\date{2022-04-07}
%\setdefaultlanguage[spelling=new]{german}

\begin{document}
	
	\begin{frame}
		\titlepage
	\end{frame}
	
	\begin{frame}

	\end{frame}

	\begin{frame}
		
	\end{frame}


	\begin{frame}
		
	\end{frame}
	
		\begin{frame}
		
	\end{frame}
	
	\begin{frame}
		
	\end{frame}
	
	
	\begin{frame}
		
	\end{frame}

		\begin{frame}
		
	\end{frame}
	
	\begin{frame}
		
	\end{frame}
	
	
	\begin{frame}
		
	\end{frame}

		\begin{frame}
	
\end{frame}

\begin{frame}
	
\end{frame}


\begin{frame}
	
\end{frame}


\end{document}